% Options for packages loaded elsewhere
\PassOptionsToPackage{unicode}{hyperref}
\PassOptionsToPackage{hyphens}{url}
%
\documentclass[
]{article}
\usepackage{amsmath,amssymb}
\usepackage{iftex}
\ifPDFTeX
  \usepackage[T1]{fontenc}
  \usepackage[utf8]{inputenc}
  \usepackage{textcomp} % provide euro and other symbols
\else % if luatex or xetex
  \usepackage{unicode-math} % this also loads fontspec
  \defaultfontfeatures{Scale=MatchLowercase}
  \defaultfontfeatures[\rmfamily]{Ligatures=TeX,Scale=1}
\fi
\usepackage{lmodern}
\ifPDFTeX\else
  % xetex/luatex font selection
\fi
% Use upquote if available, for straight quotes in verbatim environments
\IfFileExists{upquote.sty}{\usepackage{upquote}}{}
\IfFileExists{microtype.sty}{% use microtype if available
  \usepackage[]{microtype}
  \UseMicrotypeSet[protrusion]{basicmath} % disable protrusion for tt fonts
}{}
\makeatletter
\@ifundefined{KOMAClassName}{% if non-KOMA class
  \IfFileExists{parskip.sty}{%
    \usepackage{parskip}
  }{% else
    \setlength{\parindent}{0pt}
    \setlength{\parskip}{6pt plus 2pt minus 1pt}}
}{% if KOMA class
  \KOMAoptions{parskip=half}}
\makeatother
\usepackage{xcolor}
\usepackage[margin=1in]{geometry}
\usepackage{graphicx}
\makeatletter
\newsavebox\pandoc@box
\newcommand*\pandocbounded[1]{% scales image to fit in text height/width
  \sbox\pandoc@box{#1}%
  \Gscale@div\@tempa{\textheight}{\dimexpr\ht\pandoc@box+\dp\pandoc@box\relax}%
  \Gscale@div\@tempb{\linewidth}{\wd\pandoc@box}%
  \ifdim\@tempb\p@<\@tempa\p@\let\@tempa\@tempb\fi% select the smaller of both
  \ifdim\@tempa\p@<\p@\scalebox{\@tempa}{\usebox\pandoc@box}%
  \else\usebox{\pandoc@box}%
  \fi%
}
% Set default figure placement to htbp
\def\fps@figure{htbp}
\makeatother
\setlength{\emergencystretch}{3em} % prevent overfull lines
\providecommand{\tightlist}{%
  \setlength{\itemsep}{0pt}\setlength{\parskip}{0pt}}
\setcounter{secnumdepth}{-\maxdimen} % remove section numbering
\usepackage{float}
\usepackage{bookmark}
\IfFileExists{xurl.sty}{\usepackage{xurl}}{} % add URL line breaks if available
\urlstyle{same}
\hypersetup{
  pdftitle={Offense vs Defense: An Analysis in the NBA's Play Style},
  pdfauthor={Team 22},
  hidelinks,
  pdfcreator={LaTeX via pandoc}}

\title{\href{https://github.com/STAT107-F25-NBA-Analytics/STAT107-NBAAnalytics}{Offense
vs Defense: An Analysis in the NBA's Play Style}}
\author{Team 22}
\date{2025-11-07}

\begin{document}
\maketitle

\subsection{Abstract}\label{abstract}

Dunk or get dunked on --- that's the name of the game when it comes to
the NBA. You have twenty-four seconds to sink a shot or get one put down
on your hoop, and currently the rules have seemed to strongly favor
offenses and penalize physical defenses. Does that mean defense is less
useful now, or does defense still play just as crucial a part and simply
look different from what we expect? This paper seeks to investigate what
efficient defenses and offenses look like by identifying key metrics
that aggregate to describe efficient play, and ultimately analyze what
wins games more, scoring points or allowing points. To conduct this
analysis, we will develop models describing offensive and defensive
efficiency and apply nested F-tests across the models and an aggregate
to identify which contributes more to winning games. (Insert findings
later)

\subsection{Introduction}\label{introduction}

The NBA's rulebook since 2004 and ongoing regularly makes updates that
drive the game to at least seem both extremely offense-driven and
defense-second, with the latter seemingly being a mere afterthought if
not outright not integrated into defensive philosophies. This perception
of the game may have a strong sense of truth, but there is a very viable
possibility that it only describes half the game and leaves sorely
underdeveloped defensive outlook and strategy. To bring the best out of
players' technical abilities, strengthen and develop the NBA-seeking
talent pool with both the players and aspiring coaches, and help the NBA
tangibly understand how to fight now-rampant criticisms of the NBA
`going soft', it is crucial to better understand the current defensive
structure of the game and how to best develop it. By analyzing which key
metrics contribute to defensive or offensive efficiency and developing
models to understand how much each throughput affects winning outcomes,
we can create this better understanding of the flipside to the
offensive-minded game style. In order to best conduct this analysis, we
will break the analysis into two phases.

For the first phase of analysis, we will first categorize the variables
into defensive and offensive categories. Once we class the variables, we
will test for intercategorical collinearity and develop a full model
that selects one variable per collinear relationship to avoid
redundancy. Once the full model is developed, we will employ techniques
like backwards elimination to develop statistically significant
offensive and defensive models that explain a team's win within a game
accurately.

For the second phase of analysis, we will then create an aggregate model
from the two initial models that is computed as a composite. Using all
three models the, we will conduct nested F-tests to determine whether
the offensive model or defensive model provides the stronger signal
within the composite.

\subsection{Data}\label{data}

In order to conduct this analysis on NBA gameplay statistics, we are
using \href{\%22./data/nba_raw.csv\%22}{regular season data (original
data)} compiled over a time range from 2010-2024 (Korolyk, 2024). This
dataset is compiled by Vasilii Korolyk and is publicly available at
\url{https://github.com/NocturneBear/NBA-Data-2010-2024} for academic
use under its
\href{https://github.com/NocturneBear/NBA-Data-2010-2024?tab=MIT-1-ov-file}{MIT
License}. The original data contains over 33,000 observations and
documents 57 variables over each entry.

Within the raw data given, one of the variables was not a statistic and
was in fact a helpful utility called AVAILABLE\_FLAGS, which indicated
whether the data was healthy enough for use. As a result, when cleaning
the data we initially dropped all entries that didn't have a value of 1,
which indicated they were healthy. After dropping those entries, we then
removed all variables that have no bearing on the intended research or
methodologies along with fully-empty rows. After cleaning, we're left
with around 28K observations and 27 variables. The variables for the
fully cleaned data are as follows

\subsubsection{Dimensions}\label{dimensions}

\begin{itemize}
\tightlist
\item
  Season
\item
  Team ID
\item
  Team Abbreviation
\item
  Team Name
\item
  Game ID
\item
  Game Date
\item
  Matchup
\item
  Win/Loss Status
\end{itemize}

\subsubsection{Metrics}\label{metrics}

\begin{itemize}
\tightlist
\item
  Field Goals Made
\item
  Field Goals Attempted
\item
  Field Goal Accuracy
\item
  Three Pointers Made
\item
  Three Pointers Attempted
\item
  Three Pointer Accuracy
\item
  Free Throws Made
\item
  Free Throws Attempted
\item
  Free Throw Accuracy
\item
  Offensive Rebounds
\item
  Defensive Rebounds
\item
  Total Rebounds
\item
  Assists
\item
  Turnovers
\item
  Steals
\item
  Blocks
\item
  Blocks Against
\item
  Personal Fouls
\item
  Personal Fouls Drawn
\end{itemize}

Before we can use this data though, note something of key importance ---
three point shots are counted as field goals, since they are a type of
field goal along with two-point shots. Mathematically,

\[ 
\text{FG} = \text{3P} + \text{2P}
\]

From here we can derive formulas to develop two-point metrics as follows

\[ 
\text{2PA} = \text{FGA} - \text{3PA}, \text{ 2PM} = \text{FGM} - \text{3PM}, \text{ 2P\%} = \frac{\text{FGM} - \text{3PM}}{\text{FGA} - \text{3PA}}
\]

After computing these derived metrics, general field goals is
essentially rendered a composite of two available metrics, and hence
obsolete. As a result, we drop the original field goal metricsm finally
leaving us with a dataset of 28K observations and 27 variables named
like so:

\subsubsection{Dimensions}\label{dimensions-1}

\begin{itemize}
\tightlist
\item
  Season
\item
  Team ID
\item
  Team Abbreviation
\item
  Team Name
\item
  Game ID
\item
  Game Date
\item
  Matchup
\item
  Win/Loss Status
\end{itemize}

\subsubsection{Metrics}\label{metrics-1}

\begin{itemize}
\tightlist
\item
  Three Pointers Made
\item
  Three Pointers Attempted
\item
  Three Pointer Accuracy
\item
  Free Throws Made
\item
  Free Throws Attempted
\item
  Free Throw Accuracy
\item
  Offensive Rebounds
\item
  Defensive Rebounds
\item
  Total Rebounds
\item
  Assists
\item
  Turnovers
\item
  Steals
\item
  Blocks
\item
  Blocks Against
\item
  Personal Fouls
\item
  Personal Fouls Drawn
\item
  Two Pointers Made
\item
  Two Pointers Attempted
\item
  Two Pointer Accuracy
\end{itemize}

As a final preliminary to EDA and data visualization, we will also
categorize the metrics between offense and defense as follows along with
rationale:

\subsubsection{Offensive Metrics}\label{offensive-metrics}

\begin{itemize}
\tightlist
\item
  Three Pointers: Point-scoring event
\item
  Free Throws: Point-scoring event
\item
  Offensive Rebounds: Initiates offensive play via repossession on
  offensive drive
\item
  Assists: Directly enables and precedes a point-scoring event
  (two/three pointers)
\item
  Turnovers: Offensive loss of possession
\item
  Two Pointers: Point-scoring event
\item
  Personal Fouls Drawn: Enables a point-scoring event (free throw)
\item
  Blocks Against: Disrupts a point-gaining opportunity for team
\end{itemize}

\subsubsection{Defensive Metrics}\label{defensive-metrics}

\begin{itemize}
\tightlist
\item
  Defensive Rebounds: Initiates possession after defensive drive
\item
  Steals: Interrupts opponent's offensive drive and initiates possession
  from defensive play
\item
  Blocks: Disrupts a point-gaining opportunity for opponent
\item
  Personal Fouls: Penalty for illegal defensive maneuver
\end{itemize}

\subsection{EDA \& Data Visualization}\label{eda-data-visualization}

When computing models using the dataset, we must first analyze any
potential colinearities between variables to identify potential
redundancies in model construction prior to regression and pruning. To
do so, we may produce a correlation matrix and run colinearity tests on
variables of concern. When checking for colinearity, we will use the
typical threshold of

\[ |\rho| > 0.7\]

\pandocbounded{\includegraphics[keepaspectratio]{img/plots/corr_heatmap.png}}

\textbf{Figure 1.1: Correlation Matrix}

When viewing the correlation heatmap (Figure 1.1) for the dataset, we
must keep in mind the symmetry about the principal diagonal due to the
mirroredness of the graph's structure. Furthermore, cells along the
diagonal will trivially show and high colinearity due to this structure
and how certain variables are calculated. With this in mind, only
intensely-colored cells and clusters for intracategorical variables
(i.e.~three point and two point shots since both are offensive metrics)
will be taken for concern. For simple verification, we may also want to
test for colinearity for simple verification where there is a concern in
an overlap of the space --- practically and mathematically --- certain
variables are produced in (i.e.~three point and two point shots since
both are attempted during a drive as binary shot choices).

We first notice a dark cluster of cells between three-point and
two-point attempts, so we check these two metrics for colinearity.

\pandocbounded{\includegraphics[keepaspectratio]{img/plots/Three_Pointers_Attempted_vs_Two_Pointer_Attempts.png}}

\[\rho = -0.64\]

\(\rho = -0.64\) does not meet the required magnitude threshold to
qualify as sufficiently colinear, both will be accounted for when
constructing the offensive play model prior to F-testing.

Similarly, ince three pointers and two-point shots are necessarily made
during offensive drives, there is concern about colinearity between the
successful attempts since they occupy the same production space. As a
result, we will test for colinearity between the two.

\pandocbounded{\includegraphics[keepaspectratio]{img/plots/Three_Pointers_Made_vs_Two_Pointers_Made.png}}

\[\rho = -0.368\]

The plot indicates a very weak linear relation, and \(\rho\) fails the
threshold test for colinearity. As a result, both variables can
contribute to an initial model before backward elimination.

Due to the current NBA rules, personal fouls very frequently enable
opponents to shoot free throws. Due to the confoundingness in the
production of such metrics, colinearity must be tested.

\pandocbounded{\includegraphics[keepaspectratio]{img/plots/Free_Throws_Attempted_vs_Personal_Fouls_Drawn.png}}

\[\rho = 0.79\]

The plot indicates a very strong linear relation, with \(\rho\) meeting
the threshold test for colinearity. Due to this, we must choose only one
when constructing the full offensive model.

Similarly, assists necessarily precede a successful two-point attempt,
which may be concerning when considering colinearity. Due to this
concern, we will conduct a test for colinearity.

\pandocbounded{\includegraphics[keepaspectratio]{img/plots/Assists_vs_Two_Pointers_Made.png}}

\[\rho = 0.298\]

The above plot displays a very weak linear relation, and furthermore
fails the colinearity test. Consequently, both variables can contribute
to the initial offense model prior to backward elimination.

In much similar fashion to the former, assists also necessarily precede
successful three-point shots, so we move ahead with colinear analysis.

\pandocbounded{\includegraphics[keepaspectratio]{img/plots/Assists_vs_Three_Pointers_Made.png}}

\[\rho = 0.478\]

The plot indicates a very weak linear relation, and \(\rho = 0.478\)
fails the threshold test for colinearity. As a result, both metrics can
contribute to an initial model before backward elimination.

Turnovers and assists may possibly be confounding variables, especially
due to the new 2024-2025 season rules that have strengthened the
transition game and enabled very aggressive counter-offense. As a
result, we will move forward with colinear testing.

\pandocbounded{\includegraphics[keepaspectratio]{img/plots/Assists_vs_Turnover.png}}

\[\rho = -0.053\]

The plot once again indicates a very weak linear relation, and
\(\rho = -0.053\) fails the threshold test for colinearity. As a result,
both variables can be included in the offensive model before backward
elimination.

In regards to defensive concerns, personal fouls are incurred during
illegal physical contact with an offensive player. As a result, we must
test both blocks and steals against personal fouls because both
necessarily put the defensive player at risk of an illegal maneuver that
constitutes the latter.

\pandocbounded{\includegraphics[keepaspectratio]{img/plots/Blocks_vs_Personal_Fouls.png}}

\[\rho = 0.008\]

\includegraphics[width=0.75\linewidth,height=\textheight,keepaspectratio]{img/plots/Steals_vs_Personal_Fouls.png}

\[\rho = 0.028\]

Both metrics have extremely low correlation magnitudes, and as a result,
all three of these variables will be included in the full defensive
model prior to pruning.

\subsection{Analysis}\label{analysis}

Next Steps:

Full Offensive Model + Backwards Elimination

Full Defensive Model + Backwards Elimination

Develop Composite Model:

\[ C = O + M\]

Nested F-test Analysis

\subsection{Citations}\label{citations}

Korolyk, Vitalii. ``NBA Data 2010-2024 by NocturneBear.'' GitHub,
NocturneBear, 2024, github.com/NocturneBear/NBA-Data-2010-2024.

\href{https://github.com/NocturneBear/NBA-Data-2010-2024}{\pandocbounded{\includegraphics[keepaspectratio]{img/util/NocturneBear-NBA_Analytics-blue.png}}}

\end{document}
