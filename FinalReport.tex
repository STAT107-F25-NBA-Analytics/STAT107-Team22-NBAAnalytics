% Options for packages loaded elsewhere
\PassOptionsToPackage{unicode}{hyperref}
\PassOptionsToPackage{hyphens}{url}
%
\documentclass[
]{article}
\usepackage{amsmath,amssymb}
\usepackage{iftex}
\ifPDFTeX
  \usepackage[T1]{fontenc}
  \usepackage[utf8]{inputenc}
  \usepackage{textcomp} % provide euro and other symbols
\else % if luatex or xetex
  \usepackage{unicode-math} % this also loads fontspec
  \defaultfontfeatures{Scale=MatchLowercase}
  \defaultfontfeatures[\rmfamily]{Ligatures=TeX,Scale=1}
\fi
\usepackage{lmodern}
\ifPDFTeX\else
  % xetex/luatex font selection
\fi
% Use upquote if available, for straight quotes in verbatim environments
\IfFileExists{upquote.sty}{\usepackage{upquote}}{}
\IfFileExists{microtype.sty}{% use microtype if available
  \usepackage[]{microtype}
  \UseMicrotypeSet[protrusion]{basicmath} % disable protrusion for tt fonts
}{}
\makeatletter
\@ifundefined{KOMAClassName}{% if non-KOMA class
  \IfFileExists{parskip.sty}{%
    \usepackage{parskip}
  }{% else
    \setlength{\parindent}{0pt}
    \setlength{\parskip}{6pt plus 2pt minus 1pt}}
}{% if KOMA class
  \KOMAoptions{parskip=half}}
\makeatother
\usepackage{xcolor}
\usepackage[margin=1in]{geometry}
\usepackage{graphicx}
\makeatletter
\newsavebox\pandoc@box
\newcommand*\pandocbounded[1]{% scales image to fit in text height/width
  \sbox\pandoc@box{#1}%
  \Gscale@div\@tempa{\textheight}{\dimexpr\ht\pandoc@box+\dp\pandoc@box\relax}%
  \Gscale@div\@tempb{\linewidth}{\wd\pandoc@box}%
  \ifdim\@tempb\p@<\@tempa\p@\let\@tempa\@tempb\fi% select the smaller of both
  \ifdim\@tempa\p@<\p@\scalebox{\@tempa}{\usebox\pandoc@box}%
  \else\usebox{\pandoc@box}%
  \fi%
}
% Set default figure placement to htbp
\def\fps@figure{htbp}
\makeatother
\setlength{\emergencystretch}{3em} % prevent overfull lines
\providecommand{\tightlist}{%
  \setlength{\itemsep}{0pt}\setlength{\parskip}{0pt}}
\setcounter{secnumdepth}{-\maxdimen} % remove section numbering
\usepackage{float}
\usepackage{bookmark}
\IfFileExists{xurl.sty}{\usepackage{xurl}}{} % add URL line breaks if available
\urlstyle{same}
\hypersetup{
  pdftitle={Offense vs Defense: An Analysis in the NBA's Play Style},
  pdfauthor={Team 22},
  hidelinks,
  pdfcreator={LaTeX via pandoc}}

\title{\href{https://github.com/STAT107-F25-NBA-Analytics/STAT107-NBAAnalytics}{Offense
vs Defense: An Analysis in the NBA's Play Style}}
\author{Team 22}
\date{2025-11-07}

\begin{document}
\maketitle

\subsection{Abstract}\label{abstract}

Dunk or get dunked on --- that's the name of the game when it comes to
the NBA. You have twenty-four seconds to sink a shot or get one put down
on your hoop, and currently the rules have seemed to strongly favor
offenses and penalize physical defenses. Does that mean defense is less
useful now, or does defense still play just as crucial a part and simply
look different from what we expect? This paper seeks to investigate what
efficient defenses and offenses look like by identifying key metrics
that aggregate to describe efficient play, and ultimately analyze what
wins games more, scoring points or allowing points. To conduct this
analysis, we will develop models describing offensive and defensive
efficiency and apply nested F-tests across the models and an aggregate
to identify which contributes more to winning games. (Insert findings
later)

\subsection{Introduction}\label{introduction}

The NBA's rulebook since 2004 and ongoing regularly makes updates that
drive the game to at least seem both extremely offense-driven and
defense-second, with the latter seemingly being a mere afterthought if
not outright not integrated into defensive philosophies. This perception
of the game may have a strong sense of truth, but there is a very viable
possibility that it only describes half the game and leaves sorely
underdeveloped defensive outlook and strategy. To bring the best out of
players' technical abilities, strengthen and develop the NBA-seeking
talent pool with both the players and aspiring coaches, and help the NBA
tangibly understand how to fight now-rampant criticisms of the NBA
`going soft', it is crucial to better understand the current defensive
structure of the game and how to best develop it. By analyzing which key
metrics contribute to defensive or offensive efficiency and developing
models to understand how much each throughput affects winning outcomes,
we can create this better understanding of the flipside to the
offensive-minded game style. In order to best conduct this analysis, we
will break the analysis into two phases.

For the first phase of analysis, we will first categorize the variables
into defensive and offensive categories. Once we class the variables, we
will test for intercategorical collinearity and develop a full model
that selects one variable per collinear relationship to avoid
redundancy. Once the full model is developed, we will employ techniques
like backwards elimination to develop statistically significant
offensive and defensive models that explain a team's win within a game
accurately.

For the second phase of analysis, we will then create an aggregate model
from the two initial models that is computed as a composite. Using all
three models the, we will conduct nested F-tests to determine whether
the offensive model or defensive model provides the stronger signal
within the composite.

\subsection{Data}\label{data}

In order to conduct this analysis on NBA gameplay statistics, we are
using \href{\%22./data/nba_raw.csv\%22}{regular season data (original
data)} compiled over a time range from 2010-2024 (Korolyk, 2024). This
dataset is compiled by Vasilii Korolyk and is publicly available at
\url{https://github.com/NocturneBear/NBA-Data-2010-2024} for academic
use under its
\href{https://github.com/NocturneBear/NBA-Data-2010-2024?tab=MIT-1-ov-file}{MIT
License}. The original data contains over 33,000 observations and
documents 57 variables over each entry.

Within the raw data given, one of the variables was not a statistic and
was in fact a helpful utility called AVAILABLE\_FLAGS, which indicated
whether the data was healthy enough for use. As a result, when cleaning
the data we initially dropped all entries that didn't have a value of 1,
which indicated they were healthy. After dropping those entries, we then
removed all variables that have no bearing on the intended research or
methodologies along with fully-empty rows. After cleaning, we're left
with around 28K observations and 27 variables. The variables for the
fully cleaned data are as follows

\subsubsection{Dimensions}\label{dimensions}

\begin{itemize}
\tightlist
\item
  Season
\item
  Team ID
\item
  Team Abbreviation
\item
  Team Name
\item
  Game ID
\item
  Game Date
\item
  Matchup
\item
  Win/Loss Status
\end{itemize}

\subsubsection{Metrics}\label{metrics}

\begin{itemize}
\tightlist
\item
  Field Goals Made
\item
  Field Goals Attempted
\item
  Field Goal Accuracy
\item
  Three Pointers Made
\item
  Three Pointers Attempted
\item
  Three Pointer Accuracy
\item
  Free Throws Made
\item
  Free Throws Attempted
\item
  Free Throw Accuracy
\item
  Offensive Rebounds
\item
  Defensive Rebounds
\item
  Total Rebounds
\item
  Assists
\item
  Turnovers
\item
  Steals
\item
  Blocks
\item
  Blocks Against
\item
  Personal Fouls
\item
  Personal Fouls Drawn
\end{itemize}

Before we can use this data though, note something of key importance ---
three point shots are counted as field goals, since they are a type of
field goal along with two-point shots. Mathematically,

\[ 
\text{FG} = \text{3P} + \text{2P}
\]

From here we can derive formulas to develop two-point metrics as follows

\[ 
\text{2PA} = \text{FGA} - \text{3PA}, \text{ 2PM} = \text{FGM} - \text{3PM}, \text{ 2P\%} = \frac{\text{FGM} - \text{3PM}}{\text{FGA} - \text{3PA}}
\]

Furthermore, more fundamentally to gameplay, winning can't merely be
captured by shot accuracy as total shot count and shot type matter as
well. To account for this, we can develop a metric that captures
point-driven shot aggression, the three point volume. This can be
computed exactly as so

\[
\text{3PV} = \frac{\text{3PA}}{\text{FGA}}
\]

After computing these derived metrics, general field goals is rendered
fully obsolete. As a result, we drop the original field goal metrics
finally leaving us with a dataset of 28K observations and 28 variables
named like so:

\subsubsection{Dimensions}\label{dimensions-1}

\begin{itemize}
\tightlist
\item
  Season
\item
  Team ID
\item
  Team Abbreviation
\item
  Team Name
\item
  Game ID
\item
  Game Date
\item
  Matchup
\item
  Win/Loss Status
\end{itemize}

\subsubsection{Metrics}\label{metrics-1}

\begin{itemize}
\tightlist
\item
  Three Pointers Made (3PM)
\item
  Three Pointers Attempted (3PA)
\item
  Three Pointer Accuracy (3P\%)
\item
  Three Point Volume (3PV)
\item
  Free Throws Made (FTM)
\item
  Free Throws Attempted (FTA)
\item
  Free Throw Accuracy (FT\%)
\item
  Offensive Rebounds (ORB)
\item
  Defensive Rebounds (DRB)
\item
  Total Rebounds (RB)
\item
  Assists (ASS)
\item
  Turnovers (TOV)
\item
  Steals (ST)
\item
  Blocks (B)
\item
  Blocks Against (BA)
\item
  Personal Fouls (PF)
\item
  Personal Fouls Drawn (PFD)
\item
  Two Pointers Made (2PM)
\item
  Two Pointers Attempted (2PA)
\item
  Two Pointer Accuracy (2P\%)
\end{itemize}

As a final preliminary to EDA and data visualization, we will also
categorize the metrics between offense and defense as follows along with
rationale:

\subsubsection{Offensive Metrics}\label{offensive-metrics}

\begin{itemize}
\tightlist
\item
  Three Pointers: Point-scoring event
\item
  Three Point Volume: The proportion
\item
  Free Throws: Point-scoring event
\item
  Offensive Rebounds: Initiates offensive play via repossession on
  offensive drive
\item
  Assists: Directly enables and precedes a point-scoring event
  (two/three pointers)
\item
  Turnovers: Offensive loss of possession
\item
  Two Pointers: Point-scoring event
\item
  Personal Fouls Drawn: Enables a point-scoring event (free throw)
\item
  Blocks Against: Disrupts a point-gaining opportunity for team
\end{itemize}

\subsubsection{Defensive Metrics}\label{defensive-metrics}

\begin{itemize}
\tightlist
\item
  Defensive Rebounds: Initiates possession after defensive drive
\item
  Steals: Interrupts opponent's offensive drive and initiates possession
  from defensive play
\item
  Blocks: Disrupts a point-gaining opportunity for opponent
\item
  Personal Fouls: Penalty for illegal defensive maneuver
\end{itemize}

\newpage

\subsection{EDA \& Data Visualization}\label{eda-data-visualization}

To get a grasp of if and how NBA playstyle has changed across the
seasons, we can take a look at time series for certain key metrics.

\pandocbounded{\includegraphics[keepaspectratio]{img/plots/Three_Point_Volume_Time-Series.png}}

\begin{center}
\textbf{Figure 1.1: Time Series of Three Point Volume}
\end{center}

3PV is arguably the most crucial into understanding shifts in playstyle,
as it encapsulates perfectly how aggressively a team chases points.
Additionally, tt exhibits minimal confoundingness since shot selection
during possession is binary, and very few rules interfere with the
components of the metric. As seen in Figure 1.1, 3PV has seen a sharp
increase over the years, and this is about as blatant as it gets with
how aggressive teams are with scoring high rather than getting in the
paint and fighting towards the rim. Furthermore, league accuracy on the
shot has also increased as demonstrated below.

\pandocbounded{\includegraphics[keepaspectratio]{img/plots/Three_Pointer_Accuracy_Time-Series.png}}

\begin{center}
\textbf{Figure 1.2: Time Series of Three Point Accuracy}
\end{center}

We do see an anomalous inflection for the 2020-21 and 2021-22 seasons,
and this is possibly due to the impact of COVID-19 on the league. There
may be a concern that computations on data from this time may be
suspicious, but we have to acknowledge that the federal lockdown only
affected logistics and not gameplay incentive. Regardless, the general
upward trend in accuracy further reflects that three-point shooting is a
skill emphasized in-game and by extension, a significant aspect of
professional training.

\newpage

Several other shot accuracy metrics have gone up as well, albeit far
more inconsistently, which still further suggests how important the
three point shot has become to the league.

\vspace{1em}

\pandocbounded{\includegraphics[keepaspectratio]{img/plots/Two_Pointer_Accuracy_Time-Series.png}}

\begin{center}
\textbf{Figure 1.3: Time Series of Two Point Accuracy}
\end{center}

\pandocbounded{\includegraphics[keepaspectratio]{img/plots/Free_Throw_Accuracy_Time-Series.png}}

\begin{center}
\textbf{Figure 1.4: Time Series of Free Throw Accuracy}
\end{center}

\vspace{1em}

Figures 1.1-4 all strongly indicate a very shot-heavy offensive play
style, that too aggressively for points less so than battle towards the
rim and in the paint. This sort of finesse-first style has been
perceived as fans as a lack of physicality, and surprisingly the fans
are not wrong in this regard. Down below, Figure 1.5 demonstrates how
volatile and decentered offensive rebounds have become, which is
typically a strong indicator of physical aggression for reposession of
the ball after a failed shot attempt.

\pandocbounded{\includegraphics[keepaspectratio]{img/plots/Offensive_Rebounds_Time-Series.png}}

\begin{center}
\textbf{Figure 1.5: Time Series of Mean ORB Count}
\end{center}

Interestingly though, the tail-end of the graph displays an increase in
ORB count in the league and possibly indicates a new type of physical
demand far more focused on players just being able to grab the ball
after a shot bouncing off the glass. We must emphasize though that this
is far from definitive as the tail increase does not make up a
significant part of the graph and is too recent to strongly claim a
reimagined sense of physicality.

\newpage

Defensive rebounds tell an equally fascinating story of how the league
has changed as well, as 1.6 shows below.

\pandocbounded{\includegraphics[keepaspectratio]{img/plots/Defensive_Rebounds_Time-Series.png}}

\begin{center}
\textbf{Figure 1.6: Time Series of Mean DRB Count}
\end{center}

The first half of the graph shows a clear increase in DRB count for the
league, standing sharply against a one-dimensional no-contact,
offense-first narrative around modern gameplay in the league. But
starting in 2018-2019 season, there is a slow but clearly negative
change in defensive rebound count. This could be explained by harsher
rules around contact, but the argument doesn't quite fully hold water
for DRBs. That story changes fully with a different metric though ---
steals.

Steals are by far the riskiest defensive maneuver, as they can easily
result in illegal contact. It is fair to argue objectively that it is
simultaneously a high-risk, high-reward maneuver as it allows the
interruption of an opponent's drive and enables a fast break. That's not
how the league's teams see it though, and the numbers don't lie.

\pandocbounded{\includegraphics[keepaspectratio]{img/plots/Steals_Time-Series.png}}

\begin{center}
\textbf{Figure 1.7: Time Series of Mean ST Count}
\end{center}

Steal count has drastically decreased over the years, and this dramatic
decrease indicates a high fear factor for incurring a penalty, which at
best gives the opponent an opportunity to reset their drive and pace
through and at worst produces a free throw. When paired with Figure 1.4,
it understandably becomes a significant concern especially to allow a
free throw attempt to be taken.

\newpage

As a result, we naturally expect to see that offensive metrics have
stronger associations to winning. While the data does seem to
corroborate this intuition, especially when a Spearman test is applied,
some of the data visually doesn't look strongly associated despite the
Spearman test saying otherwise.

\pandocbounded{\includegraphics[keepaspectratio]{img/plots/Three_Point_Volume_Winningness-For-Props.png}}

\begin{center}
$3x^2$
\textbf{Figure 1.8: Winningness of 3PV}
\end{center}

Surprisingly, 3

\subsection{Analysis}\label{analysis}

Next Steps:

Full Offensive Model + Backwards Elimination

Full Defensive Model + Backwards Elimination

Develop Composite Model:

\[ C = O + M\]

Nested F-test Analysis

\subsection{Citations}\label{citations}

Korolyk, Vitalii. ``NBA Data 2010-2024 by NocturneBear.'' GitHub,
NocturneBear, 2024, github.com/NocturneBear/NBA-Data-2010-2024.

\href{https://github.com/NocturneBear/NBA-Data-2010-2024}{\pandocbounded{\includegraphics[keepaspectratio]{img/util/NocturneBear-NBA_Analytics-blue.png}}}

\end{document}
